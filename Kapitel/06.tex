\chapter{Schlussfolgerungen und Ausblick}

\noindent Schlussendlich lässt sich sagen, dass \acp{GAN} eine sehr beeindruckende Technologie sind, die in der Zukunft noch viel Potential hat. Die Technologie ist zwar sehr komplex und bietet viel Raum für Fehler, allerdings ist sie auch sehr vielseitig und kann in vielen Bereichen eingesetzt werden. Bereits heute werden \acp{GAN} in spannenden Projekten, wie „thispersondoesnotexist.com“ eingesetzt, welche die Möglichkeiten bereits heute greifbar machen, auch wenn es sich um eine sehr junge Technologie handelt und die Qualität sowie die Effizienz im Umgang mit Ressourcen hoffentlich in der Zukunft weiter verbessert wird. Trotz dessen wird auch \ac{GAN} keine Universallösung bleiben und ist besser als Werkzeug in einem prall gefüllten Werkzeugkasten zu verstehen, der für jeden Einsatzzweck ein entsprechendes Werkzeug bereitstellt. \\

\noindent Ein Beispiel für ein weiteres Werkzeug, das einen ähnlichen Aufgabenbereich wie \acp{GAN} abdeckt, sind Diffusionsmodelle. Diese bieten deutlich feineren Zugriff auf die Generationsparameter und arbeiten stabiler als \acp{GAN}, zum Beispiel betreffen sie Probleme wie Mode Collapse nicht. Allerdings sind diese noch rechenintensiver als \acp{GAN} und benötigen viel Fine-Tuning, um ansprechende Resultate zu erhalten, was bei \acp{GAN} die KI selbst übernimmt. Eine perfekte Methode gibt es schließlich nicht und so werden \acp{GAN} wohl auch zukünftig ein wichtiges Werkzeug im Werkzeugkasten der künstlichen Intelligenz bleiben.\\

\noindent 

\newpage
