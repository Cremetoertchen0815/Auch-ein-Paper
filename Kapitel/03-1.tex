\chapter{Generative Adversarial Networks}

\section{Konzept}

\noindent \acfp{GAN} basieren auf der zuvor genannten Technologie des unüberwachten Lernens und bedienen somit ähnliche Einsatzzwecke, nämlich der Generation von Daten, die vom menschlichen Gehirn nicht von reellen Daten unterscheiden lassen. Im Falle der \acp{GAN} geschieht dies mit der Hilfe von zwei unabhängigen, impliziten generativen Modellen: dem Generator und dem Diskriminator. Diese beiden Modelle werden als „adversarial“ (gegensätzlich) bezeichnet, da sie im Wettbewerb zueinander stehen und einander trainieren.  \\

\noindent Diese Art der künstlichen Intelligenz hat sich vor allem bezüglich der Bildverarbeitung als sehr erfolgreich erwiesen. Dies wird nicht nur in der reinen Generation von Bildern eingesetzt, sondern erlaubt ebenfalls andere bildbezogenen Methoden, wie die Super-Resolution, welche es erlaubt Bilder mit niedrigem Detailgrad neue Details zu erschaffen, die im Original nicht vorhanden sind. Eine weitere Option ist der sogenannte Style Transfer, welcher die Möglichkeit bietet den Stil eines Bildes, z.B. realistisch, abstrakt, Zeichentrick, etc. auf ein anderes Bild übertragen zu können. Besonders die Super-Resolution, auch AI-based Upscaling genannt, wird bereits heut zutage weitläufig eingesetzt, um beispielsweise Musikvideos, welche vor langer Zeit auf analoge Formate, wie Magnetbändern aufgezeichnet wurde, zu restaurieren und den analogen, 576i-Standard Definition-Look zu entfernen. Dadurch können mit genug Feinarbeit alte Videos so bearbeitet werden, dass sie nicht nur eine moderne Auflösung wie UHD besitzen, sondern auch wesentlich schärfer und klarer aussehen. \\

\noindent Um besser nachvollziehen zu können, wie die zwei konkurrierenden Modelle in solch komplexen Prozessen resultieren können, lohnt es sich hier ein genauer Blick auf die Architektur der \acp{GAN} zu werfen.

\newpage