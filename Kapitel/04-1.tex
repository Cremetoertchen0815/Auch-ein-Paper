\chapter{Anwendungen von GANs}

\noindent 
Das Ziel von GANs ist es, Modelle zu schaffen, die in der Lage sind, Daten zu generieren, die von menschlichen Beobachtern nicht von echten Daten unterschieden werden können. GANs haben Anwendungen in verschiedenen Bereichen gefunden, darunter Bildgenerierung, Stiltransfer, Super-Resolution, Generierung von realistischen Texten und mehr.

\section{Bildsynthese}


\noindent Im Bereich der Bildgenerierung werden diverse GANs bereits erfolgreich eingesetzt. Darunter sind Beispielsweise: 
\begin{itemize}
\item Artbreeder
\item StyleGAN
\item BigGAN
\item CycleGAN
\end{itemize}

\noindent Jedoch ermöglicht das antagonistische System eines GAN, nicht nur die Möglichkeit bilder zu generieren, sondern auch generierte Bilder zu erkennen. So ist es möglich zu unterscheiden ob es sich beim vorliegenden Bild um ein Original handelt oder es durch eine AI generiert wurde.
Dies hat insbesondere Relevanz, da in bezug auf, mittel künstlicher Intelligenz generierter Kunst, immer wieder die Frage des Urheberrechtschutzes im raum liegt. So ist es dank GAN möglich zu erkennen, falls urheberrechtsgeschützte Bilder als Ausgangsmaterial verwendet wurden. \\

\noindent Dadurch könnten zukünftig einige moralische und ethische Fragen der KI-Kunst geklärt werden.

\newpage
