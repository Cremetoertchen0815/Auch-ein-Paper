\section{Mode Collapse}

\noindent Ein weiteres Problem, welches bei der Verwendung von \acp{GAN} auftreten kann, ist das sogenannte \textit{Mode Collapse}, was auf Deutsch so viel wie „Modus-Kollaps“ bedeutet. Der Modus ist ein statistischer Lagewert und bezeichnet den Wert aus einer Sammlung von Werten, der am häufigsten vorkommt. Der Begriff \textit{Mode Collapse} bezeichnet dabei den Kollaps des Generators zu seinem eigenen Modus, sprich der Generator generiert nur noch eine bestimmte Art von Bildern, die sich in der Regel nur minimal voneinander unterscheiden.  Da der Generator versucht das Resultat zu produzieren, welches den Diskriminator am besten zufrieden stellt, ist ein einseitiger Ausgang nichts ungewöhnliches. Allerdings ist die Aufgabe des Diskriminators ein Ergebnis, das zu häufig vorkommt abzulehnen und zu verhindern. Dadurch muss der Generator sein Reportoire an Ausgaben erweitern, um den Diskriminator weiterhin überlisten zu können. Sollte die nächste Diskriminator-Generation allerdings in einem lokalen Minimum feststecken, kann es passieren, dass diese nicht lernt den einseitigen Ausgang abzulehnen. Dadurch wird der Generator nicht mehr dazu gezwungen sein Reportoire zu erweitern, sondern fixiert sich auf einen speziellen Diskriminator und lernt diesen zu überlisten. Der Diskriminator schafft es dann auch nicht mehr das Problem zu überwinden und stagniert. In diesem Fall ist der Generator nicht mehr in der Lage, neue Bilder zu generieren, sondern produziert nur noch ähnliche Bilder: das Modell ist kollabiert. \\

\noindent Ein Möglichkeit dieses Problem zu umgehen ist die Verwendung von alternativen Verlustfunktionen, wie die Wasserstein-Verlustfunktion, welche dem Diskriminator hilft wiederkehrende Ergebnisse abzulehnen. Eine weitere Möglichkeit ist die Verwendung von \textit{Unrolled GANs}, welche eine Verlustfunktion für den Generator verwenden, die nicht nur die Klassifikation des momentanen Diskriminators einbezieht, sondern auch die möglichen zukünftigen Diskriminator-Versionen. Dadurch, dass der Generator sich an mehrere Diskriminatoren anpassen muss, kann er sich also nicht auf einen speziellen Diskriminator versteifen und das Ergebnis bleibt variierter.  \cite{training}

\newpage
