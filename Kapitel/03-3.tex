\section{Training}

\noindent Das Training von \acp{GAN} basiert auf unüberwachtem Lernen und erfolgt in einem iterativen Prozess, der in drei Schritten durchgeführt wird: \\
\begin{enumerate}
    \item Der Generator erzeugt eine Reihe von Daten auf basis des Latent Spaces.
    \item Der Diskriminator erhält zufällig das vom Generator erzeugte Bild oder ein echtes Bild und überprüft die Echtheit.
    \item Das Ergebnis des Diskriminators wird ausgewertet und beide Netze werden durch das Resultat beeinflusst. \\
\end{enumerate}

\noindent Dabei werden beiden Modelle abwechselnd trainiert, damit die einzelnen Modelle sich besser auf das Ergebnis des jeweiligen Gegenspielers reagieren kann und sich das Ziel nicht konstant verschiebt. Der Generator wird trainiert, um den Diskriminator zu täuschen, indem er Daten erzeugt, die von echten Daten nicht zu unterscheiden sind. Der Diskriminator wird trainiert, um den Generator zu täuschen, indem er die vom Generator erzeugten Daten nicht von echten Daten unterscheiden kann. Die beiden Modelle trainieren sich schließlich gegenseitig, bis der Diskriminator circa die Hälfte der Daten nicht korrekt zuordnen kann. Die Gründe hierfür liegen in der Konvergenz des Systems und werden in Kapitel \ref{chap:FTC} näher besprochen. Zu diesem Zeitpunkt gilt das Netz als fertig trainiert\cite{training}. \\

\noindent Trotz seiner simplen Architektur ist das Training eines \acp{GAN} ein komplexer Prozess, der viel Zeit und Rechenleistung benötigt. Zusätzlich benötigt der Prozess ein großes Maß an Daten. Die rechenintensive Natur wird allerdings im Laufe der Zeit immer weniger zum Problem werden, aufgrund Moore's Law. Dieses besagt, „dass sich die Anzahl an Komponenten in einem einzigen integrierten Schaltkreis bei minimalen Kosten jedes Jahr verdoppelt.“\cite{moore} Dieser Trend wird sich zwar in den nächsten Jahren verlangsamen, dennoch ist davon auszugehen, dass sich die Leistung von Rechnern weiterhin steigern wird. Für das Problem der großen benötigten Datensätze bietet das Internet eine Lösung. Dieses ist nämlich eine nahezu unerschöpfliche Quelle an Daten. Diese Daten können dabei jedoch nicht einfach so übernommen werden, denn das Trainieren einer KI erfordert Ausgangsdaten, die vorsichtig ausgewählt und überprüft werden müssen. Dies erzeugt einen erheblichen Mehraufwand und ist ein großer einschränkender Faktor für die Entwicklungen von künstlichen Intelligenzen und somit auch \acp{GAN}.
\newpage