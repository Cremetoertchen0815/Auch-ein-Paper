\chapter{Herausforderungen}

\section{Failure to Converge}
\label{chap:FTC}

\noindent Auch wenn es sich bei \ac{GAN} um eine Technologie handelt, die sehr beeindruckende Resultate erzielen kann, so bietet sich aufgrund der komplexen Natur der Technologie auch viel Raum für Fehler und Fragilität. Einer dieser Fehler ist das sogenannte \textit{Failure to Converge}, was auf Deutsch so viel wie „Fehlschlagen der Konvergenz“ bedeutet. Dieser Fehler tritt auf, wenn die beiden Modelle nicht mehr in der Lage sind, sich gegenseitig zu trainieren und somit keine Konvergenz stattfindet. Dies liegt vor allem an den Wechselwirkungen zwischen Generator und Diskriminator. \\

\noindent Die treibende Kraft hinter dem Systems stammen ist die Rivalitätsbeziehung zwischen den beiden Modellen. Sollte allerdings eins der beiden Modelle im Vergleich zu gut werden, wird dadurch das jeweils andere Modell geschädigt. Problematisch wird das vorallem, wenn der Generator zu gut wird. Wird angenommen, dass der Generator nurnoch perfekte Resultate produziert, hat der Diskriminator keine Möglichkeit mehr ein künstliches und ein reales Bild zu unterscheiden. In diesem Fall hat der Diskriminator keine Basis mehr für die Berechnung der Wahrscheinlichkeit und muss praktisch gesehen zufällig entscheiden, schafft also statistisch nurnoch eine Erfolgsquote von 50\%. Da die Entwicklung des Generator allerdings auf inhaltsvollem Feedback des Diskriminators beruht, kann dieses sich nicht mehr weiterentwickeln. Im schlimmsten Fall wirkt das Feedback des Diskriminators schädlich auf den Generator und verringert seine Erfolgsquote wieder. Für ein \ac{GAN} ist Konvergenz also oft nur ein transitioneller statt einem stabilen Zustand. \\

\noindent Mit der Zeit wurden einige Methoden eingeführt, welche die Wahrscheinlichkeit des \textit{Failure to Converge} verringern. Diese greifen häufig in die Funktionsweise des Diskriminators ein und fügen beispielsweise Rauschen in die Eingangsdaten des Diskriminators hinzu. Dies soll dabei helfen dem Diskriminator trotz hoher Qualität der generierten Bilder das Unterscheiden der Daten zu erleichtern, wodurch durch vorsichtige Konfiguration der „Münzwurf“ verhindert werden kann.\cite{training} \\