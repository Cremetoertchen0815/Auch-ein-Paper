\section{Generative Modelle}

\noindent Generative Modelle sind Modelle, welche durch unüberwachtes Lernen erzeugt werden, einer Unterkategorie des Machine Learnings, in welche sich auch \acp{GAN} einordnen lassen. Im Gegensatz zu anderen Deep Learning Modellen, wie zum Beispiel \acfp{CNN}, welche die Daten klassisch als Input-Output-Paar oder auch Daten-Label-Paar zum Training bereitgestellt bekommen, werden bei generativen Modellen keine Labels benötigt, um die Daten zu klassifizieren. Stattdessen werden die Daten selbst analysiert und daraus ein Modell erzeugt, welches die Daten möglichst gut abbildet. Dieses Modell kann dann verwendet werden, um neue Daten zu generieren, welche den Trainingsdaten möglichst ähnlich sind.\cite{mlbook}\\

\noindent Diese Art des Machine Learnings wird in vielen Bereichen eingesetzt, wie zum Beispiel in der Bildverarbeitung, der Spracherkennung, der Sprachsynthese, der Textverarbeitung oder der Musiksynthese. Sie werden genutzt, um Daten zu generieren, die kaum von natürlichen Daten zu unterscheiden sind. \\

\noindent Generative Modelle können in zwei Kategorien unterteilt werden: \textit{explicit generative models} und \textit{implicit generative models}. Bei \textit{explicit generative models} wird die Wahrscheinlichkeitsverteilung der Daten explizit modelliert. Dies geschieht in der Regel durch die Verwendung von Bayes'schen Netzen. Bei \textit{implicit generative models} wird die Wahrscheinlichkeitsverteilung der Daten nicht explizit modelliert, sondern durch ein Modell approximiert. Dies wird in der Regel mit neuronalen Netzen realisiert, welche die Wahrscheinlichkeitsverteilung der Daten approximieren. Bei \acp{GAN} handelt es sich um ein \textit{implicit generative model}, welches mit neuronalen Netzen arbeitet. In Kapitel 3 wird dabei die Architektur von \acp{GAN} genauer erläutert.\\


\newpage